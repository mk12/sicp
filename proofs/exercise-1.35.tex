\documentclass{article}

\title{Golden Ratio Fixed Point}
\author{Mitchell Kember}
\date{26 May 2014}

\usepackage{amsmath}
\usepackage{amssymb}
\usepackage{commath}

\newcommand*{\p}{\ensuremath{\varphi}}

\begin{document}
\maketitle

\noindent Exercise 1.13 of \emph{Structure and Interpretation of Computer Programs} asks us to
\begin{quote}
Show that the golden ratio \p{} (Section 1.2.2) is a fixed point of the transformation \(x\mapsto 1+1/x\), and use this fact to compute \p{} by means of the \texttt{fixed-point} procedure.
\end{quote}

\section*{Definitions}
The golden ratio \p{} is the positive real number that satisfies
\[\p = 1 + \frac{1}{\p}.\]
Multiplying by \p{}, we get another property of the golden ratio:
\[\p^2 = \p + 1.\]

\section*{Proof}
We would like to prove that each transformation \(x\mapsto 1+1/x\) causes the value of \(x\) to become closer to \p{}. Given an initial guess \(x>1\), let \(y=1+1/x\) be the improved guess. We must prove that \(\abs{y-\p} < \,\abs{x-\p}\). To begin, we will simplify the left-hand side of the inequality:

\begin{align*}
\abs{y-\p}
&= \abs{1+\frac1x-\p}\\
&= \abs{\frac{x+1-\p x}{x}}\\
&= \abs{\frac{x+1-(1+1/\p)x}{x}}\\
&= \abs{\frac{x+1-x-x/\p}{x}}\\
&= \abs{\frac{\p-x}{x\p}}.
\end{align*}

The error of the improved guess, \(\abs{y-\p}\), is related to the error of the orignal guess, \(\abs{x-\p}\). We can rearrange a bit more to make this clear:
\[\abs{y-\p} = \abs{\frac{\p-x}{x\p}} = \frac{\abs{x-\p}}{\abs{x\p}}.\]
To prove that \(\abs{y-\p}<\abs{x-\p}\), we must show that \(\abs{x\p}>1\), because dividing the original error by a number greater than one will produce a new error smaller than the original. We already stipulated that \(x>1\), and we know the value of the golden ratio is \(\p=(1+\sqrt{5})/2\), therefore \(x\p>1\) and \(\abs{x\p}>1\).

Since \(\abs{xp}>1\) and \(\abs{y-\p} = \abs{x-\p}/\abs{xp}\), we have
\[\abs{y-\p}<\abs{x-\p},\]
therefore each transformation \(x\mapsto y\) brings \(x\) closer to \p{}, and consequently \p{} is a fixed point of the transformation \(x\mapsto 1+1/x\). \(\blacksquare\)


\end{document}
