\documentclass{article}

\title{Fibonacci and the Golden Ratio}
\author{Mitchell Kember}
\date{18 May 2014}

\usepackage{amsmath}
\usepackage{amssymb}
\usepackage{multicol}
\usepackage{commath}

\begin{document}
\maketitle

\noindent Exercise 1.13 of \emph{Structure and Interpretation of Computer Programs} asks us to
\begin{quote}
Prove that Fib(\(n\)) is the closest integer to \(\varphi^n/\sqrt{5}\), where \(\varphi=(1+\sqrt{5})/2\). Hint: Let \(\psi=(1-\sqrt{5})/2\). Use induction to prove that Fib(\(n)=(\varphi^n-\psi^n)/\sqrt{5}\).
\end{quote}

\section*{Definitions}

The constants \(\varphi\) and \(\psi\) are the positive and negative solutions to the golden ratio equation for a rectangle with side lengths of 1 and \(x\): \[\frac1x=\frac{x}{1+x}.\] Both \(\varphi\) and \(\psi\) satisfy the following properties: \[1+x=x^2 \qquad \textup{and} \qquad \frac1x + 1 = x.\]

The Fibonacci sequence begins with 0 and 1; each subsequent element is the sum of the two elements preceding it: \[0, 1, 1, 2, 3, 5, 8, 13, 21, \dots.\] Using the Fib function, we can define the sequence recursively with
\begin{align*}
\textup{Fib}(0) &= 0;\\
\textup{Fib}(1) &= 1;\\
\textup{Fib}(n) &= \textup{Fib}(n-1) + \textup{Fib}(n-2).
\end{align*}

\section*{Proof}

We will begin by proving by induction that \begin{equation}\label{eqn:phi-psi}\textup{Fib}(n)=\frac{\varphi^n-\psi^n}{\sqrt{5}}.\end{equation}
First, we will demonstrate that equation~\eqref{eqn:phi-psi} is true for the three base cases: Fib(0), Fib(1), and Fib(2).
When \(n=0\), \(\textup{LS}=\textup{Fib}(0)=0\) and
\begin{align*}
\textup{RS} &= \frac{\varphi^0-\psi^0}{\sqrt{5}}\\
&= \frac{1-1}{\sqrt5}\\
&= 0.
\end{align*}
When \(n=1\), \(\textup{LS}=\textup{Fib}(1)=1\) and
\begin{align*}
\textup{RS} &=  \frac{\varphi^1-\psi^1}{\sqrt{5}}\\
&= \frac{\frac{1+\sqrt5}{2}-\frac{1-\sqrt5}{2}}{\sqrt5}\\
&= \frac{\frac{2\sqrt5}{2}}{\sqrt5}\\
&= 1.
\end{align*}
When \(n=2\), \(\textup{LS}=\textup{Fib}(2)=1\) and
\begin{align*}
\textup{RS} &=  \frac{\varphi^2-\psi^2}{\sqrt{5}}\\
&= \frac{\left(\frac{1+\sqrt5}{2}\right)^2-
\left(\frac{1-\sqrt5}{2}\right)^2}{\sqrt5}\\
&= \frac{\frac{(1+\sqrt5)^2-(1-\sqrt5)^2}{4}}{\sqrt5}\\
&= \frac{\left(\left(1+\sqrt5\right) - \left(1-\sqrt5\right)\right)\left(\left(1+\sqrt5\right)+\left(1-\sqrt5\right)\right)}{4\sqrt5}\\
&= \frac{\left(2\sqrt5\right)\left(2\right)}{4\sqrt5}\\
&= 1.
\end{align*}

Now comes the inductive step. For equation~\eqref{eqn:phi-psi} to be true for the entire sequence, we must be able to prove that \[\textup{Fib}(n) = \textup{Fib}(n-1) + \textup{Fib}(n-2)\] using equation~\eqref{eqn:phi-psi} as the definiton of Fib. We know that \(\textup{LS}=(\varphi^n-\psi^n)/\sqrt{5}\). We can show that the right-hand side is equal:
\begin{align*}
\textup{RS} &= \frac{\varphi^{n-1}-\psi^{n-1}}{\sqrt{5}} + \frac{\varphi^{n-2}-\psi^{n-2}}{\sqrt{5}}\\
&= \frac{\left(\varphi^{n-1}+\varphi^{n-2}\right)-\left(\psi^{n-1}+\psi^{n-2}\right)}{\sqrt5}\\
&= \frac{\varphi^n\left(\varphi^{-1}+\varphi^{-2}\right)-\psi^n\left(\psi^{-1}+\psi^{-2}\right)}{\sqrt5}\\
&= \frac{\varphi^n\varphi^{-1}\left(1+\varphi^{-1}\right)-\psi^n\psi^{-1}\left(1+\psi^{-1}\right)}{\sqrt5}\\
&= \frac{\varphi^n\varphi^{-1}\left(\varphi\right)-\psi^n\psi^{-1}\left(\psi\right)}{\sqrt5}\\
&= \frac{\varphi^n-\psi^n}{\sqrt5}.
\end{align*}

Now we must show that Fib(\(n\)) is the closest integer to \(\varphi^n/\sqrt{5}\). The absolute difference between these two values must remain less than or equal to one half for the latter to round to the former:
\begin{align*}
\abs{\textup{Fib}(n)-\frac{\varphi^n}{\sqrt5}} &\le \frac12\\
\abs{\frac{\varphi^n-\psi^n}{\sqrt5} - \frac{\varphi^n}{\sqrt5}} &\le \frac12\\
\abs{-\frac{\psi^n}{\sqrt5}} &\le \frac12\\
\frac{\abs{-\psi^n}}{\sqrt5} &\le \frac12\\
\abs{\psi^n} &\le \frac{\sqrt5}{2}.
\end{align*}
The value of \(\psi\) is about \(-0.618\), therefore its absolute value is about \(0.618\). For all nonnegative values of \(n\), \(\abs{\psi^n}\le1\). The value of \(\sqrt5/2\) is about 1.12, which is greater than 1. Therefore, the equation above is true and  \(\varphi^n/\sqrt{5}\) will always round to Fib(\(n\)). \(\blacksquare\)
\end{document}
