\startcomponent exercise-1.35
\product proofs

\define\p{\math{\varphi}}

\starttext

\startproblem
Show that the golden ratio \p{} (Section 1.2.2) is a fixed point of the
transformation $x\mapsto 1+1/x$, and use this fact to compute \p{} by means of
the \code{fixed-point} procedure.
\stopproblem
The golden ratio \p{} is the positive real number that satisfies
\startformula
\p = 1 + \frac{1}{\p}.
\stopformula
Multiplying by \p{}, we get another property of the golden ratio:
\startformula
\p^2 = \p + 1.
\stopformula
For \p{} to be a fixed point, each repeated transformation $x\mapsto 1+1/x$ must
cause the value of $x$ to become closer to \p{}.

\starttheorem
Given any $x>1$ and $y=1/1+x$, $y$ is closer than $x$ to \p{}, meaning $x$ and
$y$ satisfy $\abs{y-\p} < \,\abs{x-\p}$.
\stoptheorem

\startproof
To begin, we will simplify the left-hand side of the inequality:
\startformula\startalign
\NC \abs{y-\p} \NC = \abs{1+\frac1x-\p} \NR
\NC \NC = \abs{\frac{x+1-\p x}{x}} \NR
\NC \NC = \abs{\frac{x+1-(1+1/\p)x}{x}} \NR
\NC \NC = \abs{\frac{x+1-x-x/\p}{x}} \NR
\NC \NC = \abs{\frac{\p-x}{x\p}} \NR
\NC \NC = \frac{\abs{x-\p}}{\abs{x\p}}.
\stopalign\stopformula
We already stipulated that $x>1$, and we know the value of the golden ratio is
$\p=(1+\sqrt{5})/2 \approx 1.618 > 1$, therefore $x\p>1$ and $\abs{x\p}>1$.

Since $\abs{xp}>1$ and $\abs{y-\p} = \abs{x-\p}/\abs{xp}$, we have
\startformula
\abs{y-\p}<\abs{x-\p},
\stopformula
therefore $y$ is closer than $x$ to \p{}, and consequently \p{} is a fixed point
of the transformation $x\mapsto 1+1/x$.
\stopproof

\stoptext
