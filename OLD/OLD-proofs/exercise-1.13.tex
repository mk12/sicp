\startcomponent exercise-1.13
\product proofs

\starttext

\startproblem
Prove that $\Fib(n)$ is the closest integer to $\varphi^n/\sqrt5$, where
$\varphi=(1+\sqrt5)/2$. Hint: Let $\psi=(1-\sqrt5)/2$. Use
induction to prove that $\Fib(n)=(\varphi^n-\psi^n)/\sqrt5$.
\stopproblem
The constants $\varphi$ and $\psi$ are the positive and negative solutions to the
golden ratio equation for a rectangle with side lengths of 1 and $x$:
\startformula
\frac{1}{x}=\frac{x}{1+x}.
\stopformula
Both $\varphi$ and $\psi$ satisfy two properties:
\startformula
1+x=x^2 \quad \text{and} \quad \frac{1}{x} + 1 = x, \quad \forall x \in
\{\varphi,\psi\}.
\stopformula
The Fibonacci sequence begins with 0 and 1; each subsequent element is the sum
of the two elements preceding it:
\startformula
0, 1, 1, 2, 3, 5, 8, 13, 21, \dots.
\stopformula
Using the Fib function, we can define the sequence recursively with
\startformula\startalign
\NC \Fib(0) \NC = 0, \NR
\NC \Fib(1) \NC = 1, \NR
\NC \Fib(n) \NC = \Fib(n-1) + \Fib(n-2).
\stopalign\stopformula

\startlemma
The exact value of $\Fib(n)$ is given by
\placeformula[eq:fib]
\startformula
\Fib(n)=\frac{\varphi^n-\psi^n}{\sqrt5}.
\stopformula
\stoplemma

\startproof
First, we will demonstrate that \in{equation}[eq:fib] is true for the three
base cases: $\Fib(0)$, $\Fib(1)$, and $\Fib(2)$. When $n=0$,
$\text{LS}=\Fib(0)=0$ and
\startformula
\text{RS}
= \frac{\varphi^0-\psi^0}{\sqrt5}
= \frac{1-1}{\sqrt5} \NR
= 0.
\stopformula
When $n=1$, $\text{LS}=\Fib(1)=1$ and
\startformula
\text{RS}
= \frac{\varphi^1-\psi^1}{\sqrt5}
= \frac{\frac{1+\sqrt5}{2}-\frac{1-\sqrt5}{2}}{\sqrt5}
= \frac{\frac{2\sqrt5}{2}}{\sqrt5}
= 1.
\stopformula
When $n=2$, $\text{LS}=\Fib(2)=1$ and
\startformula\startalign
\NC \text{RS} \NC = \frac{\varphi^2-\psi^2}{\sqrt5} \NR
\NC \NC = \frac{\left(\frac{1+\sqrt5}{2}\right)^2
          - \left(\frac{1-\sqrt5}{2}\right)^2}{\sqrt5} \NR
\NC \NC = \frac{\frac{(1+\sqrt5)^2-(1-\sqrt5)^2}{4}}{\sqrt5} \NR
\NC \NC = \frac{\left(\left(1+\sqrt5\right)-\left(1-\sqrt5\right)\right)
          \left(\left(1+\sqrt5\right)+\left(1-\sqrt5\right)\right)}{4\sqrt5} \NR
\NC \NC = \frac{\left(2\sqrt5\right)(2)}{4\sqrt5} \NR
\NC \NC = 1.
\stopalign\stopformula

Now comes the inductive step. For \in{equation}[eq:fib] to be true for the
entire sequence, we must be able to prove that
\startformula
\Fib(n) = \Fib(n-1) + \Fib(n-2)
\stopformula
using \in{equation}[eq:fib] as the definiton of Fib. We know that
$\text{LS}=(\varphi^n-\psi^n)/\sqrt5$. We can show that the right-hand side is
equal:
\startformula\startalign
\NC \text{RS} \NC = \frac{\varphi^{n-1}-\psi^{n-1}}{\sqrt5}
                    + \frac{\varphi^{n-2}-\psi^{n-2}}{\sqrt{5}} \NR
\NC \NC = \frac{\left(\varphi^{n-1}+\varphi^{n-2}\right)
          - \left(\psi^{n-1}+\psi^{n-2}\right)}{\sqrt5} \NR
\NC \NC = \frac{\varphi^n\left(\varphi^{-1}+\varphi^{-2}\right)
          - \psi^n\left(\psi^{-1}+\psi^{-2}\right)}{\sqrt5} \NR
\NC \NC = \frac{\varphi^n\varphi^{-1}\left(1+\varphi^{-1}\right)
          - \psi^n\psi^{-1}\left(1+\psi^{-1}\right)}{\sqrt5} \NR
\NC \NC = \frac{\varphi^n\varphi^{-1}\left(\varphi\right)
          - \psi^n\psi^{-1}\left(\psi\right)}{\sqrt5} \NR
\NC \NC = \frac{\varphi^n-\psi^n}{\sqrt5}.
\stopalign\stopformula
This completes the proof of \in{equation}[eq:fib].
\stopproof

\starttheorem
$\Fib(n)$ is the closest integer to $\varphi^n/\sqrt5$, where
$\varphi=(1+\sqrt5)/2$.
\stoptheorem

\startproof
The absolute difference between these two values must remain less than or equal
to one half for the latter to round to the former:
\placeformula[eq:round]
\startformula\startalign
\NC \abs{\Fib(n)-\frac{\varphi^n}{\sqrt5}} \NC \le \frac12 \NR[+]
\NC \abs{\frac{\varphi^n-\psi^n}{\sqrt5} - \frac{\varphi^n}{\sqrt5}} \NC
    \le \frac12 \NR
\NC \abs{-\frac{\psi^n}{\sqrt5}} \NC \le \frac12 \NR
\NC \frac{\abs{-\psi^n}}{\sqrt5} \NC \le \frac12 \NR
\NC \abs{\psi^n} \NC \le \frac{\sqrt5}{2}.
\stopalign\stopformula
The value of $\psi$ is about $-0.618$, therefore its absolute value is about
0.618. For all nonnegative values of $n$, $\abs{\psi^n}\le1$. The value of
$\sqrt5/2$ is about 1.12, which is greater than 1. Therefore,
\in{equation}[eq:round] is true and $\varphi^n/\sqrt5$ will always round to
$\Fib(n)$.
\stopproof

\stoptext
\stopcomponent
