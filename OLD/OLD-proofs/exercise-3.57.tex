\startcomponent exercise-3.57
\product proofs

\define[1]\placeholder{\m{\langle}#1\m{\rangle}}

\starttext

\startproblem
How many additions are performed when we compute the $n$th Fibonacci number
using the definition of \code{fibs} based on the \code{add-streams} procedure.
Show that the number of additions would be exponentially greater if we had
implemented \code{(delay \placeholder{exp})} simply as \code{(lambda ()
\placeholder{exp})}, without using the optimization provided by the
\code{memo-proc} procedure described in Section 3.5.1.
\stopproblem
Let $A(n)$ represent the number of additions required to compute $\Fib(n)$ using
\code{fibs}. The sequence begins with $\Fib(0)=0$ and $\Fib(1)=1$.

\starttheorem
Using \code{fibs} and the optimized implementation of \code{delay}, the number
of additions grows as $\Theta(n)$; specifically,
\placeformula[eqn:mem]
\startformula
A(0) = 0 \quad \text{and} \quad A(n) = n-1, \; n > 1.
\stopformula
\stoptheorem

\startproof
Since 0 and 1 form the base case of the procedure, they require no additions and
therefore $A(0)=A(1)=0$. To get the next element, we add the first element of
the sequence, 0, to the first element of the \code{cdr} of the sequence, 1. We
have $\Fib(2) = 0 + 1 = 1$ and $A(2) = 1$. In general,
\placeformula[eqn:rec]
\startformula
A(n) = A(n-1) + 1,
\stopformula
because the next element is obtained by doing one more addition. 

\in{Equation}[eqn:mem] correctly produces zero for $n = 0$ and for $n = 1$. We
can show that it is also true for all $n>1$ by mathematical induction. If it is
true, we should be able to verify \in{equation}[eqn:rec] by substituting $n-1$
for $A(n)$. We have $\text{LS} = A(n) = n - 1$, and
\startformula
\text{RS} = A(n-1) + 1 = ((n-1)-1) + 1 = n - 1 = \text{LS},
\stopformula
therefore \in{equation}[eqn:mem] is true for all $n≥0$.
\stopproof

\starttheorem
Using \code{fibs} and the implementation of \code{delay} that does \emph{not}
use memoization, the number of additions grows as $\Theta(e^n)$; specifically,
\placeformula[eqn:unmem]
\startformula
A(n) = \frac14\left(\left(1-\sqrt2\right)^{n}
       + \left(1+\sqrt2\right)^{n}
       - 2\right).
\stopformula
\stoptheorem

\startproof
As before, $A(0) = A(1) = 0$, and \in{equation}[eqn:unmem] satisfies both of
these. To get the next element, we add 0 to 1, so $A(2) = 1$. To compute
$\Fib(3)$, we must repeat the one addition of $\Fib(2)$, repeat it again to
extend the \code{cdr} of the sequence one more element, giving us $\Fib(3) = 1 +
1 = 2$ and $A(3) = 3$. In general,
\placeformula[eqn:recun]
\startformula
A(n) = A(n-2) + 2A(n-1) + 1.
\stopformula
Here is why: to get the next element, we must repeat the work of getting the
previous element in the stream: $A(n-1)$. We must do it again to extend the
\code{cdr} of the sequence to that same element, because it will be the addend.
We must then compute $\Fib(n-2)$ to extend the other stream going into
\code{add-streams}, because this element will be the augend. Finally, we add the
addend to the augend, and this is the last addition.

We can once again use induction to prove \in{equation}[eqn:unmem] for all $n>1$.
Let $a = 1-\sqrt2$ and let $b = 1+\sqrt2$. Substituing it into
\in{equation}[eqn:recun], we have
\startformula
\text{LS} = \frac14\left(a^n + b^n - 2\right),
\stopformula
and on the other side, we have
\startformula\startalign
\NC \text{RS} = \NC A(n-2) + 2A(n-1) + 1 \NR
\NC = \NC \frac14\left(a^{n-2} + b^{n-2} - 2\right)
          + 2\cdot\frac14\left(a^{n-1} + b^{n-1} - 2\right) + 1\NR
\NC = \NC \frac14a^{n-2} +\frac14b^{n-2} - \frac12
          + \frac12a^{n-1} +\frac12b^{n-1} - 1 + 1 \NR
\NC = \NC \frac14\left(a^{n-2}+b^{n-2}+2a^{n-1}+2b^{n-1}-2\right) \NR
\NC = \NC \frac14\left(a^na^{-2} + b^nb^{-2}+2a^na^{-1}+2b^nb^{-1}-2\right) \NR
\NC = \NC \frac14\left(a^n\left(a^{-2} + 2a^{-1}\right)
                     + b^n\left(b^{-2} + 2b^{-1}\right) - 2\right) \NR
\NC = \NC \frac14\left(a^n\left(\frac1{a^2}+\frac2a\right)
                     + b^n\left(\frac1{b^2}+\frac2b\right) - 2\right) \NR
\NC = \NC \frac14\left(a^n\left(\frac{1+2a}{a^2}\right)
                     + b^n\left(\frac{1+2b}{b^2}\right) - 2\right) \NR
\NC = \NC \frac14\left(a^n\left(\frac{1+2\left(1-\sqrt2\right)}
                                     {\left(1-\sqrt2\right)^2}\right)
                     + b^n\left(\frac{1+2\left(1+\sqrt2\right)}
		                     {\left(1+\sqrt2\right)^2}\right)
	             - 2\right) \NR
\NC = \NC \frac14\left(a^n\left(\frac{3-2\sqrt2}{3-2\sqrt2}\right)
                     + b^n\left(\frac{3+2\sqrt2}{3+2\sqrt2}\right)
	             - 2\right) \NR
\NC = \NC \frac14\left(a^n + b^n - 2\right) \NR
\NC = \NC \text{LS},
\stopalign\stopformula
therefore \in{equation}[eqn:unmem] is true for all $n≥0$.
\stopproof

\stoptext
\stopcomponent
